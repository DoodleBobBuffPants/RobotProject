\documentclass[letterpaper, 10 pt, conference]{ieeeconf}  % Comment this line out if you need a4paper
%\documentclass[a4paper, 10pt, conference]{ieeeconf}   % Use this line for a4 paper
\IEEEoverridecommandlockouts                                    % This command is only needed if you want to use the \thanks command
\overrideIEEEmargins                                                 % Needed to meet printer requirements.

%In case you encounter the following error:
%Error 1010 The PDF file may be corrupt (unable to open PDF file) OR
%Error 1000 An error occurred while parsing a contents stream. Unable to analyze the PDF file.
%This is a known problem with pdfLaTeX conversion filter. The file cannot be opened with acrobat reader
%Please use one of the alternatives below to circumvent this error by uncommenting one or the other
%\pdfobjcompresslevel=0
%\pdfminorversion=4

% See the \addtolength command later in the file to balance the column lengths
% on the last page of the document

\usepackage{graphics}    % for pdf, bitmapped graphics files
%\usepackage{epsfig}    % for postscript graphics files
\usepackage{mathptmx} % assumes new font selection scheme installed
\usepackage{times}        % assumes new font selection scheme installed
\usepackage{amsmath}  % assumes amsmath package installed
\usepackage{amssymb}  % assumes amsmath package installed

\title{\LARGE \bf Formation Control}

\author{Ajay Ahir, Ben Philps and Sumaiyah Kola}

\begin{document}

\maketitle
\thispagestyle{empty}
\pagestyle{empty}

%%%%%%%%%%%%%%%%%%%%%%%%%%%%%%%%%%%%%%%%%%%%%%%%%%%%%%%%%%%%%%%%%%%%%%%%%%%%%%%%
\begin{abstract}

TODO:

\begin{itemize}
\item Only the bare minimum of words have been used so far, each section needs more detail and more math.
\end{itemize}

\end{abstract}

%%%%%%%%%%%%%%%%%%%%%%%%%%%%%%%%%%%%%%%%%%%%%%%%%%%%%%%%%%%%%%%%%%%%%%%%%%%%%%%%
\section{INTRODUCTION}

Robots may need to navigate in formation, perhaps as part of a display or to guarantee 360$^{\circ}$ coverage. However, staying in formation while navigating may not be possible. Take tight corridors for example. A wide diamond formation may not fit, thus preventing the robots from fulfilling their purpose. The problem here is finding a way to split and merge the robots, so that they can navigate around obstacles and through limited space to get back into formation. \\

A range of formations and environments are considered, and both a centralized and decentralized form of the problem is taken into account. \\

TODO: Need to explain experimental setup here - ros/gazebo, multi robot launching, creation of worlds

\section{APPROACH}

In order to solve the problem of formation control, we took the approach outlined in this$^{\cite{c1}}$ paper which defines behaviors for our robots. Each behavior generates a motor schema - a velocity for the robot. These velocities are: formation velocity, goal velocity, and static/dynamic obstacle avoidance velocity. Finally, the resulting velocity vectors are weighted and summed to give the overall robot behavior. \\

\subsection{Formation Velocity}

First we defined the formations we would like to move in, and these are listed below:

\begin{itemize}
\item Line - The robots move through space side by side.
\item Column - The robots move through space one after the other.
\item Diamond - The robots move in a square-like form
\item Wedge - The robots move in a distorted line, similar to the shape of an arrowhead or boomerang.
\end{itemize}~\\

TODO: Vector of relative positions for each robot. \\

Two methods for keeping the robots in formation were considered. The first is called unit reference, where the average position taken across all robots is used to define the point around which the formation is held. Each robot then generates a velocity to an offset from this position, which defines their place in the formation as given by a robot ID preassigned to each of them. Whilst this solution is straightforward to implement in a centralized fashion, its major disadvantage is that in a decentralized implementation every robot is required to know the position of every other robot in order to find the formation center. \\

The approach we took is called leader-follower. With this, a certain robot is designated the leader, which we chose based on their ID and predefined this for each formation. The leader does not need to calculate a formation velocity, as it is always considered to be in formation. The other `follower' robots then use the leader's position to generate the velocity to their position, which is an offset based on their ID as defined in the unit reference method. When decentralized, the robots only need to learn the leader's position which is achieved via message passing. \\

TODO: Define decentralized message passing links
TODO: Define `zones' (dead, control) and how we calculate velocities and desired\_position

\subsection{Goal Velocity}

With the leader-follower approach to formation control, only the leader needs a goal velocity, as the followers' goal is defined by their position in the formation, which is the only place they need to aim to be. Our leader runs the RRT* algorithm once to compute an obstacle free path to a goal. \\

TODO: Brief explanation of RRT* and how the vector to the path is generated

\subsection{Static/Dynamic Obstacle Avoidance Velocity}

TODO: Update this first paragraph with details on the leader's obstacle avoidance \\

Our implementation of RRT* has built-in obstacle avoidance, so the leader does not need to a compute velocity vector for static obstacles. Each follower robot however uses laser sensors to obtain measurements that determine its distance to obstacles within a 180$^{\circ}$ cone in front of it. These measurements are fed into a rule based obstacle avoidance method which steers the robot away from static obstacles such as walls. \\

The dynamic obstacles are other robots, and the laser sensors are quite poor at detecting other robots which is why a separate implementation is required. Our implementation of this uses the position of each robot to produce a weight vector instead of a velocity vector. This weight acts as an `on-off' switch which stops a robot from moving if it is too close to another robot. We stop the robot with the lowest ID, as this reduces the combinatorial complexity to $O(n^{2})$ as we only need to consider each possible pair of robots. While there are edge cases to this approach, the weak sensor measurements should still activate the obstacle avoidance to allow that robot that is not left stationary to move.

\subsection{Combining Velocities}

Each velocity component is combined with a weight. This weight affects the nature of the robots movement. The weights we used were $0.8$ for the goal velocity, $1.2$ for the formation velocity, and $3.0$ for the static obstacle avoidance velocity. The resulting behavior forces the robots to stay in formation when there is free space, with the added effect of the followers playing `catch-up' with the leader to avoid being left behind. In the presence of obstacles, the robots will split to avoid the obstacles and merge once they are free again. The vector of dynamic obstacle avoidance weights will stop robots that get too close from moving, and the robots also stop moving once they are at their goal to avoid other velocity components from leading to erroneous and spurious movements. \\

TODO: Perhaps some pseudocode
TODO: Need to add that these are [x,y] velocities later turned into [u,w]

\subsection{Centralized and Decentralized solution}

The centralized solution uses the known position of each robot, called the `ground truth', to compute their velocity vectors in $(x,y)$ position coordinates. \\

The decentralized solution is a modified version of the centralized solution that is only designed to run for a single robot. The implementation is then ran on each robot. What it does is maintain a belief of the leader's position which is updated via message passing and limited by the connection constraints on the formation. With just the leader's position, the desired velocity to get to the goal can be computed.

TODO: Might need more of an explanation, or none if this is explained throughout

\section{RESULTS}

TODO - talk about map, error etc.

\section{CONCLUSION}

TODO - success, robustness (completeness), future work

\subsection{Figures and Tables}

Positioning Figures and Tables: Place figures and tables at the top and bottom of columns. Avoid placing them in the middle of columns. Large figures and tables may span across both columns. Figure captions should be below the figures; table heads should appear above the tables. Insert figures and tables after they are cited in the text. Use the abbreviation Fig. 1, even at the beginning of a sentence.

\begin{table}[h]
\caption{An Example of a Table}
\label{table_example}
\begin{center}
\begin{tabular}{|c||c|}
\hline
One & Two\\
\hline
Three & Four\\
\hline
\end{tabular}
\end{center}
\end{table}

\begin{figure}[thpb]
	\centering
	\framebox{\parbox{3in}{We suggest that you use a text box to insert a graphic (which is ideally a 300 dpi TIFF or EPS file, with all
		fonts embedded) because, in a document, this method is somewhat more stable than directly inserting a picture.}}
	\caption{Inductance of oscillation winding on amorphous magnetic core versus DC bias magnetic field}
	\label{figurelabel}
\end{figure}

\addtolength{\textheight}{-12cm}   % This command serves to balance the column lengths
                                               % on the last page of the document manually. It shortens
                                               % the textheight of the last page by a suitable amount.
                                               % This command does not take effect until the next page
                                               % so it should come on the page before the last. Make
                                               % sure that you do not shorten the textheight too much.

%%%%%%%%%%%%%%%%%%%%%%%%%%%%%%%%%%%%%%%%%%%%%%%%%%%%%%%%%%%%%%%%%%%%%%%%%%%%%%%%

\section{ACKNOWLEDGMENT}

\begin{itemize}
\item Ajay Ahir - 
\item Ben Philps - 
\item Sumaiyah Kola - 
\end{itemize}

%%%%%%%%%%%%%%%%%%%%%%%%%%%%%%%%%%%%%%%%%%%%%%%%%%%%%%%%%%%%%%%%%%%%%%%%%%%%%%%%

\begin{thebibliography}{99}

\bibitem{c1} T. Balch and R. C. Arkin. Behavior-based Formation Control for Multi-robot Teams. IEEE Transactions on Robotics and Automation, 1999.

\end{thebibliography}

\end{document}
